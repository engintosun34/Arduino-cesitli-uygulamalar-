\documentclass[conference]{IEEEtran}
\IEEEoverridecommandlockouts
% The preceding line is only needed to identify funding in the first footnote. If that is unneeded, please comment it out.
\usepackage{cite}
\usepackage{amsmath,amssymb,amsfonts}
\usepackage{algorithmic}
\usepackage{graphicx}
\usepackage{textcomp}
\usepackage{xcolor}
\usepackage{ragged2e}
\usepackage{indentfirst}
\usepackage{multicol}
\def\BibTeX{{\rm B\kern-.05em{\sc i\kern-.025em b}\kern-.08em
    T\kern-.1667em\lower.7ex\hbox{E}\kern-.125emX}}
\begin{document}

\title{Akıllı Ev Sistemi Projesi\\
}

\author{\IEEEauthorblockN{\large Engin Tosun}
\IEEEauthorblockA{\textit{} \\
\textit{\large200202028}\\
}
\and
\IEEEauthorblockN{\large Özgür Aydın}
\IEEEauthorblockA{\textit{} \\
\textit{\large190202087}\\
}
}

\maketitle

\section{Özet}
Bu rapor Programlama Laboratuvarı II Dersinin 2. Projesini açıklamak ve sunumunu gerçekleştirmek amacıyla oluşturulmuştur. Bu proje C dilinde Code:Blocks ortamında geliştirilmiştir.Raporda projenin tanımı, özet, yöntem,karşılaşılan sorunlar ve çözümler,sözde kod, sonuç bölümünden oluşmaktadır. Proje aşamasında yararlanılan kaynaklar raporun son bölümünde bulunmaktadır. 


\section{Proje Tanımı}
 Bu projenin amacı, Arduino üzerinde çalışan bir akıllı ev simülasyonu yapmaktır.\\

Nesnelerin İnterneti (IoT) uygulamalarının yaygınlaşması ile insanların nesneler ile olan 
iletişiminin yanı sıra nesnelerin nesneler ile olan iletişimi gün geçtikçe önem arz etmekte ve bu alandaki çalışmalar artmaktadır. Bu çalışmalardan birisi Akıllı Ev Sistemleri’dir.\\
Ev ortamında gerçekleştirilen faaliyetleri kolaylaştıran, güvenilir bir ortam sağlayan ve insan 
hayatına konfor, rahatlık veren ev otomasyonu sistemlerine Akıllı Ev denilmektedir.\\
Akıllı ev, ev teknolojileri endüstrinin birçok alanında kullanılan kontrol sistemlerinin gündelik 
hayata uyarlanması; ev otomasyonu ise bu teknolojilerin kişiye özel ihtiyaç ve isteklerine 
uygulanmasıdır. Akıllı ev tanımı, bütün bu teknolojiler sayesinde ev sakinlerinin ihtiyaçlarına 
cevap verebilen, onların hayatlarını kolaylaştıran ve daha güvenli daha konforlu ve daha 
tasarruflu bir yaşam sunan evler için kullanılmaktadır. Akıllı evler, otomatik fonksiyonları ve 
sistemleri kullanıcı tarafından uzaktan kontrol edilebilen cihazları içerirler.\\

Akıllı ev sistemlerinde bulunabilecek bazı özellikler şu şekildedir:
• Otomatik ısı sabitleme,\\
• Odalarda ışık kontrolü,\\
• Perdelerin açılıp kapanma kontrolü,\\
• Garaj kapısı kontrolü,\\
• Hırsız alarm sistemi,\\
• Ev ile ilgili bilgilerin telefondan otomatik alınması,\\
• Otomatik toprak sulama sistemi, vb.\\

\\
Projede Yapılması İstenen İsterler Hakkında:\\
Projede Proteus programında Arduino kartı kullanarak akıllı ev sistemi oluşturmamız 
beklenmektedir. Sistem içerisinde;
• Yangın alarmı, \\
• Hareket algılayan ışık sistemi,\\ 
• Dijital termometre,\\
• Kilit sistemi bulunmalıdır. \\ 
Aşağıda verilen sensör ve elemanları kullanarak belirtilen isterleri gerçekleştiriniz.\\
1. Arduino kartı olarak Arduino Mega kullanılmalıdır.\\

Projede yangın sensörü ve buzzer kullanılmalıdır. Yangın tespit edildiğinde alarm çalması sağlanmalıdır.\\
Projede hareket sensörü ve lamba kullanılmalıdır. Hareket tespit edildiğinde lamba yanması 
sağlanmalıdır.\\
rojede sıcaklık sensörü ve LCD ekran kullanılmalıdır. Algılanan sıcaklığın devamlı olarak 
LCD ekranda gösterilmesi sağlanmalıdır. Sıcaklık 20 C’nin altına düştüğünde ekrana 
“Sıcaklık düştü”, 30 C’nin üstüne çıktığında “Sıcaklık yükseldi” yazdırılmalıdır.\\
Projede tuş takımı (keypad), kırmızı ve yeşil led kullanılmalıdır. Keypad ile girilecek 4 
haneli bir şifre belirlenmelidir. Şifre yanlış girildiğinde kırmızı, doğru girildiğinde yeşil 
ledin yanması sağlanmalıdır.\\
KISITLAR: \\
• Proje Arduino IDE ve Proteus programları kullanılarak geliştirilecektir. \\
• Belirtilen bütün sensörlerin kullanılması zorunludur.\\
• Projeye başka sensörler eklenmemelidir.\\



\section{Yöntem}
Projede istenen isterler için arduino ve proteus hakkında araştırma yapıldı ve programlama için arduinonun syntaxına bakıldı ve koda uygulandı.Proteus kullanılarak tek tek her sistemin devresi internetten yardım alınarak kuruldu.
Arduino ve sensörlerin libraryleri internetten indirilerek proteus ortamına kuruldu. Sonrasında arduino kodu oluşturulmaya başlandı. Çıkan çeşitli hatalar internette araştırılarak kütüphaneleri arduino idesine kuruldu. Bütün sistemlerin çalışması kontrol edildi.\\
\begin{flushleft}
\textbf{Projede Yapılması istenen isterler hakkında\\}	
\end{flushleft}

A) Yangın alarmı buzzerın ötmesi ile gerçekleşen sistem.\\
B) Sıcaklık ölçer lcd ekrana sıcaklık değerini yazdırma ve 20 derece altı sıcaklık düştü ve 30 derece üstü sıcaklık yüksek ikazı verdirme \\
C) Kilit sistemi girilen şifreye göre kırmızı yanlış şifre yeşil doğru şifre olduğunu belirtmesi için yanması.\\
D)Hareket sensörü hareket algılayınca lambanının yanması.\\
\\






\section{Sözde Kod}
\begin{verbatim}
Yangın alarmı sözde kodu:\\
define FlamePin A0 yangın sensörü 
pini girişi\\
define buzzer 1 buzzer pini girişi\\
setup()   // kurulum
   pinMode(FlamePin, INPUT); 
   pinMode(buzzer, OUTPUT);
   gecikme(10);  
dongü() { // calisma mekanizmasi

int Flame = digitalRead(FlamePin);
if(Flame==HIGH) Yangın high verirse
  digitalWrite(buzzer, HIGH); buzzerın 
  çalışması uyarılır.
 delay(10);
else
 digitalWrite(buzzer, LOW); low verirse 
 buzzer çalışmaz.


Hareket algılayan ışık sistemi sözde kodu:\\

int kalibrasyonZamani = 10;
Sensorun hazirlanmasi 
icin verdigim zaman 
long unsigned int lowIn; 
tüm hareketin durduğunu 
varsaymadan önce sensörün milisaniye 
miktarının düşük olması gerekir
long unsigned int pause = 5000; 
boolean lockLow = true; 
boolean gecikme; 
int pirPin = 3; digital pine bağlantı 
pir sensoru icin
int lampPin = 13; //led 13 e baaglı

kurulum() kurulumu
  pinMode(pirPin, INPUT); 
  pinMode(lampPin, OUTPUT); 
  digitalWrite(pirPin, LOW); sensore kalibre icin zaman
  seriyazdır("Sensor Hazirlaniyor "); 
  for(int i = 0; i <kalibrasyonZamani ; i++)
    yazdır("."); 
 gecikme(1000); 
  yazdır(" done"); 
  yazdır("SENSOR AKTIF"); 
  gecikme(50); 

döngü()
   
  if(digitalRead(pirPin) == HIGH)    
        
    digitalWrite(lampPin, HIGH) led, sensörlerin 
    çıkış pin durumunu görselleştirir
    
    if(lockLow)      
     daha fazla çıktı yapılmadan önce DÜŞÜK'e geçişi \\
    beklediğimizden emin olur        
      lockLow = false; 
      yazdır("---");        
      yazdır("hareket algilandi= "); 
      yazdır(millis()/1000); 
      yazdır(" sn"); 
      delay(50);      
    
    gecikme = true; 
  
    
  if(digitalRead(pirPin) == LOW)    
        
    digitalWrite(lampPin, LOW); //led, sensörlerin çıkış 
    pin durumunu görselleştirir
    
    if(gecikme)      
            
 lowIn = millis() yüksekten DÜŞÜK'e geçiş 
 zamanından tasarruf edin     
 gecikme = false bunun yalnızca DÜŞÜK aşamanın 
başlangıcında yapıldığından emin olun     
sensör verilen duraklamadan daha uzun süre düşükse, \\
daha fazla hareket olmayacağını varsayıyoruz   
    
if(!lockLow && millis() - lowIn > pause)          
bu kod bloğunun yalnızca 
yeni bir hareket dizisi \\
algılandıktan sonra yeniden
yürütülmesini sağlar\\
      lockLow = true; \\
      yazdır("motion ended at "); //çıktı        
      yazdır((millis() - pause)/1000);          
      yazdır(" sec"); delay(50);       
  

Dijital termometre sözde kodu:

     
const int rs = 2, en = 3, d4 = 4, 
d5 = 5, d6 = 6, d7 = 7; lcd bağlantıları

float gecici;  lm35 icin tmp deger

kurulum() 
  analogReference(2);
  lcd.begin (16,2);  burada yazdıgım 
  kodlar programıin 
  bir kere calismasi icindir
  en üst satır(0, 0);
  lcd.yazdır(" Thermometer ");      
  lcd bir alt satıra geç(0, 1);
  delay(400);
  lcd.yzdır(" V2.0  ");
  delay (700);
  lcd.temizle(); system kurumu

void loop() 
  gecici = analogRead(A0);                                                    
  temp = temp * 0.48828125;
  temp=temp*(5.0/1023.0)*100; 
  sicaklik hesabi formulu
  gecici=gecici*1100/(1024*10);
  lcd yazdır("TEMP: ");
  lcd yazdır(gecici);
  lcd yazdır("*C");
  lcd. bir alt satıra geç(0, 1);
  if(gecici<20) 20 dereceden düşükse
    lcd yazdır("Sicaklik dustu");
  
  if(30<gecici) 30 dereceden yüksekse
  
    lcd yazdır("Sicaklik yukseldi"); 
  
  gecikme(500);
  lcd.temizle(); system kurulumu
  
 Kilit sistemi sözde kodu: \\
 
char keys[4][3]={  keypadi char 
dizisi ile tanımlama

byte rowPin[4]={6,7,8,9};
byte colPin[3]={3,4,5};

String password = "258"; password.
int position = 0;

int wrong = 0 Yanlış girdileri 
hesaplamak için değişken.


redPin = 10; pin girişleri
greenPin = 11;

Keypad keypad=Keypad(makeKeymap(keys),rowPin,colPin,4,3);
 KLAVYE HARİTASI.

int total = 0;Yanlış denemelerin sayısını 
belirlemek için değişken.
kurulum()
{
 pinMode(redPin,OUTPUT);
 pinMode(greenPin,OUTPUT);
 pinMode(buzzer, OUTPUT);
 
lcd kurulumu
}
void loop()
{
  lcd.clear();
  lcd.print(" Sifreyi Giriniz: ");
  delay(100);
  
 char pressed=keypad.getKey();
 String key[3];
  
 if(pressed)
 {
  lcd.clear();
  lcd.print(" Sifreyi Giriniz: ");
  lcd.setCursor(position,2);
  lcd.print(pressed);
  delay(500);
    if(pressed == '*' || pressed == '#') sifre giriş algoritması
          position = 0;
          setLocked(true);
          lcd.clear();
      
    else if(pressed == password[position])
          key[position]=pressed;
          position++;
     
    else if (pressed != password[position] )
          wrong++;
          position ++;
    
    if(position == 3){
          if( wrong >0)
         
               hatalı sifre wrong++
          
    else if(position == 3 && wrong == 0)
            {
        position = 0;
        wrong = 0;
        lcd.clear();
        lcd.setCursor(0,1);
        lcd.yazdır("Hos geldiniz!");
        lcd.setCursor(5,2);
        lcd.yazdır(" Kilit acildi.");
      gecikme(2000);
        
void setLocked(int locked)
  {
    if (locked)
      {
        kırmızı led yak
        yeşil led söndür
      }
    else
      {
       kırmızı led söndür
       yeşil led yak
          gecikme(2000);
           kırmızı led yak
        yeşil led söndür
         
\end{verbatim}
\\ \\
\\
\\
\\
\section{Sonuç}


Yangın Alarmı Pasif Konum\\ \\
    \includegraphics[width=9 cm,height=8 cm]{Yangın Sensörü İlk Konum.png}\\\\ \\
    \\ \\ \\
   
    Yangın Alarmı Aktif Konum\\ \\
   \includegraphics[width=9 cm,height=8 cm]{Yangın2.png}\\\\ \\ 
    
   
   Hareket Algılayan Işık Sistemi Devresi\\ \\
     \includegraphics[width=9 cm,height=8 cm]{Hareketseen.PNG}\\ \\ \\ \\ \\
     \\ \\
     
     Hareket Algılayan Işık Sistemi Devresi(Çalışır Durumda)\\ \\
     \includegraphics[width=9 cm,height=8 cm]{b1.png}\\ 
    \\
    \\
    \\
    \\ \\ \\
     
     
     
     Dijital Termometre Sistemi (Çalışır Halde)\\ 
     
      \includegraphics[width=9 cm,height=8 cm]{Termometre1.PNG}\\ \\ \\
      \\ \\
      \\
     
      
     Kilit Sistemi Devresi\\ \\
      \includegraphics[width=10 cm,height=9 cm]{c12.PNG}\\ \\
     \\ \\ \\
     
     
     
        \section{Kaynakça}
Proteus devre elemanları için;\\
- https://www.theengineeringprojects.com/2016/01/pir-sensor-library-proteus.html\\
- en.wikipedia.org/wiki/\\
- https://www.theengineeringprojects.com/2016/01/pir-sensor-library-proteus.html\\
Genel Sorunlar için;\\
-stackoverflow.com\\
-theprogrammershangout.com\\
 LaTeX Raporu hazırlamak için gerekli ekipman ve bilgiler;\\
- www.overleaf.com\\\\





\end{document}
